\documentclass{article}
\usepackage{amsthm}
\usepackage{enumerate}

\theoremstyle{definition}
\newtheorem{definition}{Definition}
\newtheorem{question}{Question}
\newtheorem{axiom}{Axiom}
\newtheorem{theorem}{Theorem}
\newtheorem{crit-theorem}{*Theorem}
\theoremstyle{definition}
\newtheorem{example}{Example}
\newtheorem{problem}{Problem}
\theoremstyle{plain}

\title{\textbf{\textsf{Linear Point Set Theory}}}
\date{}

\begin{document}
\maketitle

\section*{Order Properties}

\paragraph{Undefined Terms:} The word ``point'' and the expression ``the point
$x$ precedes the point $y$'' will not be defined.  This undefined expression
will be written $x < y$.

In these notes, all sets have at least one element.  By ``point $x$ follows the
point $y$'' we mean $y < x$.  By $x \leq y$ we mean that $x < y$ or $x = y$.

\begin{axiom}
  $S$ is a collection of points such that
  \begin{enumerate}[a.]
    \item if $x$ and $y$ are different points, then $x < y$ or $y < x$.
    \item if the point $x$ precedes the point $y$, then $x$ is not equal to $y$.
    \item if $x$, $y$, and $z$ are points such that $x < y$ and $y < z$, then $x < z$.
  \end{enumerate}

\end{axiom}

In case a misunderstanding persists concerning the relationship between
``point'', ``point set'', and $S$, let it be clear that, unless there is a
stipulation to the contrary, that

\begin{itemize}
  \item ``point'' always means a member of $S$.
  \item ``point set'' means a nonempty set of points in $S$ or a nonempty subset of $S$.
  \item $S$ is the universal set, that is, the set of all points.
\end{itemize}

\begin{example}
  Suppose by ``point'' we mean any integer and by ``$x$ precedes $y$'' we mean
  the normally understood ``$x$ is less than $y$.''  Clearly Axiom 1 is
  satisfied.
\end{example}

\begin{example}
  Suppose by ``point'' we mean a number depicted below and by ``$x$ precedes
  $y$'' we mean that $x$ is to the left sf $y$ on the number line.
\end{example}

\begin{example}
  Suppose by ``point'' we mean a number depicted below and by ``$x$ precedes
  $y$'' we mean that $x$ is to the left of $y$ on the number line.  See below:
  \[3 \qquad -2 \qquad 1 \qquad 5 \qquad -7.1 \qquad 7 \qquad 6\]
  Is this a model of Axiom 1?
\end{example}

\begin{problem}
  Think of a particular dictionary of English words.  Suppose by ``point'' we
  mean any word in that dictionary.  What interpretation could you give to
  ``precedes'' so that Axiom 1 is satisfied?
\end{problem}

\begin{problem}
  Consider the HFD Corporation, the world's largest corporation that hires
  thousands of people every year.  Suppose by ``point'' we mean an employee of
  HFD.  By ``$x$ precedes $y$'' we mean that employee $x$ was hired on at HFD
  before employee $y$.  Are Axioms 1a, 1b, and 1c all satisfied? Explain.
\end{problem}

\begin{problem}
  Find an interpretation for Axiom 1 such that Axiom 1a and 1c are satisfied,
  but 1b is not satisfied.
\end{problem}

\begin{problem}
  Give a model of Axiom 1 where ``point'' means integer and ``$x$ precedes
  $y$'' is defined in such a way that 0 precedes all other integers.
\end{problem}

When we have found meanings for the undefined terms of our axiom system such
that the axioms of our system are satisfied as we did in Examples 1 and 2
above, we have a \emph{model} of the axiom system.

\begin{problem}
  Give a meaning sf ``precedes'' where ``point'' means an ordered number pair
  and Axiom 1 is satisfied.  State what a typical region would look like.
\end{problem}

\begin{theorem}
  If $x$ and $y$ are different points such that $x < y$, then $y \not< x$.
\end{theorem}

\begin{problem}
  Consider the diagram below.

  Suppose by ``point'' we mean a point on the diagram.  By ``$x$ precedes $y$''
  we mean that the point $x$ on the diagram is to the left of the point $y$ on
  the diagram.  What parts of Axiom 1, if any, are not satisfied?
\end{problem}

\begin{problem}
  Can Theorem 1 be proved without Axiom 1c?  Explain.
\end{problem}

\begin{definition}
  If $c$ is a point of a point set $M$ such that no point of $M$ precedes $c$,
  then $c$ is called a \emph{first point} of $M$.  Define \emph{last point}
  similarly.
\end{definition}

\begin{problem}
  Does the model of Example 2 have a first point?  A last point? Explain.
\end{problem}

\begin{theorem}
  No point set has two first (last) points.
\end{theorem}

\begin{theorem}
  If $M$ is a finite set then $M$ has a first point and a last point.
\end{theorem}

\begin{theorem}
  If $M$ is a set consisting sf $n$ members, then $M$ can be written as $\{x_1,
  x_2, \cdots, x_n\}$ where $x_1 < x_2 < \cdots < x_n$.
\end{theorem}

\begin{problem}
  Give a model for the present axiom system such that $S$ has a first point and
  a last point.
\end{problem}

\begin{axiom}
  $S$ has no first point and no last point.
\end{axiom}

\begin{definition}
  If $x < z$ and $z < y$, then $z$ is said to be \textbf{between} $x$ and $y$.
  Moreover, when $x < y$ and there is a point between $x$ and $y$, the point
  set consisting of all points between $x$ and $y$ is called a \textbf{region}
  and will be referred to as the \textbf{region} $xy$.
\end{definition}

\begin{theorem}
  If $x$ is a point, then there is a region containing $x$.
\end{theorem}

\begin{question}
  Is Theorem 5 equivalent to Axiom 2?
\end{question}

\begin{problem}
  Suppose ``point'' means a point on the unit circle.  Show that there is a
  meaning of ``precedes'' such that Axioms 1 and 2 are true.
\end{problem}

\begin{theorem}
  If $x$ belongs to each of two regions $R_1$ and $R_2$ then there is a region
  $R_3$ containing $x$ that is a subset of each of $R_1$ and $R_2$.
\end{theorem}

\begin{theorem}
  If the point $x$ belongs to each member of a finite collection of regions,
  then there is some region containing $x$ which is a subset of each member of
  the collection.
\end{theorem}

\begin{definition}
  Two point sets are said to be \textbf{mutually exclusive} or
  \textbf{disjoint} if they have no points in common.  If $G$ is a collection
  of the point sets such that each two of them are mutually exclusive, then the
  sets of $G$ are said to be \textbf{pairwise mutually exclusive}.
\end{definition}

\begin{theorem}
  If $x$ and $y$ are different points, then there are two disjoint regions
  $R_x$ and $R_y$ where $R_x$ contains $x$ and $R_y$ contains $y$.
\end{theorem}

\section*{Limit Points}

\begin{definition}
  A point $p$ is said to be a \textbf{limit point} of a point set $M$ if every
  region containing $p$ contains a point $M$ distinct from $p$.
\end{definition}

\begin{problem}
  Complete The statement ``a point $p$ is not a limit point of a point set
  $M$'' means \ldots
\end{problem}

\begin{theorem}
  If the point set $H$ is a subset of the point set $K$ and $p$ is a limit
  point of $H$, then $p$ is a limit point of $K$.
\end{theorem}

\begin{theorem}
  If the point $p$ is a limit point of $H \cup K$ where $H$ and $K$ are point
  sets, then $p$ is a limit point of $H$ or $p$ is a limit point of $K$.
\end{theorem}

\begin{theorem}
  If $p$ is the limit point of the union of a finite collection $G$ of point
  sets, then $p$ is a limit point of at least one member of $G$.
\end{theorem}

\begin{theorem}
  No finite point set has a limit point.
\end{theorem}

\begin{definition}
  A point set is said to be \textbf{infinite} if it is not finite.
\end{definition}

\begin{problem}
  Prove that $p$ is a limit point of the given set $M$ in each example below.

  \begin{enumerate}[a.]
    \item $p = 0$ and $M = (0,1]$.
    \item $p = 0.5$ and $M = [0,1]$.
    \item $p = 1$ and $M$ is the set of all real numbers.
    \item $p = 0$ and $M = \{1, 1/2, 1/3, \cdots \}$.
  \end{enumerate}

\end{problem}

\begin{problem}
  Using the numbers as the model, which of the following are true and which are
  not true?
  \begin{enumerate}[a.]
    \item Some finite point set has a limit point.
    \item Any infinite number set $M$ has a limit point.
    \item Any subset of $(a,b)$ has a limit point.
    \item All limit points of a given set belong to the set.
  \end{enumerate}
\end{problem}

\begin{problem}
  Suppose there is a smallest positive number.

  \begin{enumerate}[a.]
    \item Prove that the set of all real numbers has no limit point.
    \item Prove that the square root of 2 does not exist.
    \item Prove that there are no irrational numbers.
  \end{enumerate}
\end{problem}

\begin{problem}
  Let $A = \{x : x \textnormal{ is less than zero} \}$ and $B = \{ x : x
  \textnormal{ is greater than or equal to 1} \}$.  By ``point'' we mean a member of
  $A$ or $B$. By ``precedes'' we mean the natural order of numbers.  Prove that
  1 is a limit point of $A$.
\end{problem}

\begin{theorem}
  If the point $p$ is a limit point of $H \cup K$ where $H$ is a point set and
  $K$ is a finite point set, then $p$ is a limit point of $H$.
\end{theorem}

\begin{theorem}
  If $p$ is a limit point of the point set $M$, then each region containing $p$
  contains infinitely many points of $M$.
\end{theorem}

\section*{Infinite Sequences}

The following definition is supplied for completeness.  The reader will find
that the precise definition of infinite sequence, unlike the other definitions
in these notes, does not supply much illumination on the concept.  It works
just as well if one just assumes some primitive or informal notion of infinite
sequence or list which never ends.

\begin{definition}
  An \textbf{infinite sequence} $p$ in $S$ is a function with domain the
  positive integers and range a subset of $S$. If $n$ is a positive integer,
  then $p_n$ denotes the second member of an ordered pair whose first member is
  $n$.  Moreover, we denote the sequence $p$ as $p_1, p_2, p_3, \cdots$. That
  is, we think of $p$ as a subset of $S$ with each positive integer matched to
  exactly one point in $p$.
\end{definition}

\begin{definition}
  Suppose $p_1, p_2, p_3, \cdots$ is an infinite sequence of points. Then a
  \textbf{k-tail} of $p_1, p_2, p_3, \cdots$ is the sequence $p_k, p_{k+1},
  p_{k+2}, \cdots$ where $k$ is some positive integer.
\end{definition}

\begin{definition}
  An infinite sequence of points $p_1, p_2, p_3, \cdots$ is said to
  \textbf{converge} to a point $L$ if and only if each region $R$ containing
  $L$ contains all but finitely many terms of the infinite sequence.
\end{definition}

\begin{question}
  \begin{enumerate}[a.]
    \item Does every infinite sequence have a first or a last point?
    \item Does every bounded infinite sequence have a first or a last point?
    \item Does every convergent infinite sequence have a first or a last point?
  \end{enumerate}
\end{question}

\begin{definition}
  An infinite sequence of points $p_1, p_2, p_3, \cdots$ is said to be a
  \textbf{non-decreasing} sequence if $p_n \leq p_{n+1}$ for each positive
  integer $n$.  \textbf{Nondecreasing} is defined similarly.  An infinite
  sequence is said to be \textbf{monotonic} if and only if it is nonincreasing
  or it is nondecreasing.  The infinite sequence of points $p_1, p_2, p_3,
  \cdots$ is said to be \textbf{increasing} if $p_n < p_{n+1}$ for each
  positive integer $n$.  \textbf{Decreasing} is defined similarly.
\end{definition}

\begin{question}
  Is it true that if $f$ is an infinite sequence the nthere is a monotonic
  subsequence $g$ and a monotonic subsequence $h$ such that $f = g \cup h$?
\end{question}

\begin{theorem}
  No sequence of points converges to each of two different points.
\end{theorem}

\begin{theorem}
  If $p_1, p_2, p_3, \cdots$ is an infinite sequence of distinct points
  converging to the point $L$, then $L$ is the only limit point of the set
  \{$p_1, p_2, p_3, \cdots$\}.
\end{theorem}

\begin{theorem}
  If $p_1, p_2, p_3, \cdots$ is an infinite sequence of points convergent to a
  point $L$, then some region contains the entire infinite sequence.
\end{theorem}

\section*{Closed, Closure, and Open}

\begin{definition}
  The statement that a point set $M$ is \textbf{closed} means that if $p$ is a
  limit point of $M$, then $p$ belongs to $M$.
\end{definition}

\paragraph{Notation.}  Suppose $H$ is a point set.  If $H$ has a limit point,
then $H'$ denotes the set of all limit points of $H$.  Also $\bar{H}$
(read ``H-bar'') denotes $H$ together with all of its limit points and is
called the \textbf{closure} of $H$ (not to be confused with ``closed'' below).
Note that $\bar{H} = H \cup H'$.

\begin{problem}
  Develop some conjectures concerning $\bar{H}$ and $H'$.
\end{problem}

\begin{problem}
  Define ``boundary point of a set'' and create some theorems concerning your
  definition.
\end{problem}

\begin{problem}
  Suppose $M$ is a point set.  Prove or disprove that
  \begin{enumerate}[a.]
    \item $M$ is closed if and only if $M' \subset M$.
    \item $M$ is close if and only if $\bar{M} = M$.
    \item $(\bar{M})' = \bar{M}$.
  \end{enumerate}
\end{problem}

\begin{theorem}
  If $H$ is a point set, then $\bar{H}$ is closed.
\end{theorem}

\begin{definition}
  A point set $H$ is said to be \textbf{open} if for every point $x$ of $H$
  there is a region $R$ containing $x$ such that $R$ is a subset of $H$.
\end{definition}

\begin{theorem}
  Every region is open.
\end{theorem}

\begin{problem}
  Give an example of a number set that is not open and not closed.
\end{problem}

\begin{theorem}
  If $x$ is a point, then the set of all points that precede $x$ is an open
  set.  Moreover, if $x$ is a point, then the set of all points that follow $x$
  is an open set.
\end{theorem}

\begin{definition}
  If $x < y$, then the set consisting of $x$ and $y$ together with all the
  points between $x$ and $y$ is called the \textbf{interval} $xy$.
\end{definition}

\begin{theorem}
  Every interval is closed.
\end{theorem}

\begin{theorem}
  If the open set $M$ is a proper subset of $S$, then $S -
  M$ is closed.
\end{theorem}

\begin{theorem}
  If the closed set $M$ is a proper subset of $S$, then $S
  - M$ is open.
\end{theorem}

\begin{problem}
  Give a model of Axioms 1 and 2 in which there is a proper subset that is both
  open and closed.
\end{problem}

\begin{theorem}
  If $G$ is a collection of open sets, the $\cup G$ is open.
\end{theorem}

\begin{theorem}
  If $G$ is a collection of closed sets having a common point, then $\cap G$ is
  closed.
\end{theorem}

\begin{theorem}
  If $G$ is a finite collection of open sets having a common point, then $\cap
  G$ is open.
\end{theorem}

\begin{theorem}
  If $G$ is a finite collection of closed sets, then $\cup G$ is closed.
\end{theorem}

\begin{problem}
  Does there exist a subset of the real numbers that contains no interval, is
  closed, and each point of the set is a limit point of the set?
\end{problem}

\section*{Connected Point Sets}

\begin{definition}
  Two point sets are said to be \textbf{mutually separated} if they have no
  point in common and neither of them contains a limit point of the other.
\end{definition}

\begin{definition}
  A point set is said to be \textbf{connected} if it is not the union of two
  mutually separated point sets.
\end{definition}

\begin{theorem}
  If $H$ and $K$ are two mutually separated point sets and $M$ is a connected
  subset of $H \cup K$, the $M$ is a subset of $H$ or $M$ is a subset of $K$.
\end{theorem}

\begin{theorem}
  If $G$ is a collection of connected point sets having a common point, then
  $\cup G$ is connected.
\end{theorem}

\begin{theorem}
  If $M$ is a connected point set, then $\bar{M}$ is
  connected.
\end{theorem}

\begin{theorem}
  If $M$ is a connected point set containing more than one member, then every
  point of $M$ is a limit point of $M$.
\end{theorem}

\begin{theorem}
  If $x$ and $y$ are two distinct points of a connected
  point set $M$, then the interval $xy$ is a subset of $M$.
\end{theorem}

\begin{theorem}
  If the point set $M$ is not connected, then $M$ is the
  union of two mutually separated sets $H$ and $K$ such that every point of $H$
  precedes every point of $K$.
\end{theorem}

\begin{axiom}
  \label{ax:connected}
  $S$ is connected.
\end{axiom}

\begin{theorem}
  If $H$ and $K$ are two point sets such that \mbox{$H \cup K =
  S$} and every point of $H$ precedes every point of $K$, then either $H$ has a
  last point or $K$ has a first point.  Furthermore, if $H$ has a last point,
  then $K$ does not have a first point.
\end{theorem}

\begin{theorem}
  Each region is connected.
\end{theorem}

\begin{theorem}
  Each interval is connected.
\end{theorem}

\begin{theorem}
  No proper subset of $S$ is both open and closed.
\end{theorem}

\begin{question}
  Is each of the above three theorems equivalent to Axiom 3?
\end{question}

\begin{theorem}
  If $x$ and $y$ are two points, then there is a point between them.
\end{theorem}

\begin{question}
  Is Theorem 38 equivalent to Axiom 3?
\end{question}

\begin{theorem}
  Every point of a region $R$ is a limit point of $R$.
\end{theorem}

\begin{theorem}
  Each end point of a region $R$ is a limit point of $R$.
\end{theorem}

\begin{theorem}
  No region has a first (or last) point.
\end{theorem}

\begin{theorem}
  $S$ contains infinitely many points and every point is a limit point of $S$.
\end{theorem}

\section*{Least Upper Bounds}

\begin{definition}
  A point set $M$ is said to have an \textbf{upper bound} if there is some
  point $U$ (called an upper bound of $M$) such that $x \leq U$ for each $x$ in
  $M$.  \textbf{Lower bound} is defined similarly.
\end{definition}

\begin{definition}
  The statement that $L$ is a \textbf{least upper bound} of the point set $M$
  means that $L$ is an upper bound of $M$ and no upper bound of $M$ precedes
  $L$.  \textbf{Greatest lower bound} is defined similarly.
\end{definition}

\begin{theorem}
  If each of $L_1$ and $L_2$ is a least upper bound of the point set $M$, then
  $L_1 = L_2$.  A similar statement is true for greatest lower bounds.
\end{theorem}

\begin{theorem}
  If $L$ is the least upper bound of a point set $M$ that does not contain $L$,
  then $L$ is a limit point of $M$.
\end{theorem}

\begin{theorem}
  If $M$ is a point set with an upper bound, then $M$ has a least upper bound.
  Also, if $M$ is a point set with a lower bound, then $M$ has a greatest lower
  bound.
\end{theorem}

\begin{definition}
  An infinite sequence of points $p_1, p_2, p_3, \cdots$ is said to an
  \textbf{increasing} sequence if $p_n < p_{n+1}$ for each positive integer
  $n$.
\end{definition}

\begin{theorem}
  Every bounded increasing sequence of points converges.
\end{theorem}

\begin{definition}
  A collection $G$ of sets is said to \textbf{cover} a point set $M$ if every
  point of $M$ is in at least one set of $G$, that is, $M$ is a subset of $\cup
  G$.  If some subcollection $G'$ of $G$ covers $M$, then $G'$ is said to be a
  \textbf{subcover} of $G$.  If $G$ is a collection of regions, then $G$ is
  said to be a \textbf{region cover}.  If $G$ is a collection of open sets,
  then $G$ is said to be an \textbf{open cover}.
\end{definition}

\begin{theorem}
  A point set has a region cover if and only if it has an open cover.
\end{theorem}

\begin{theorem}
  If $G$ is a collection of regions where each point of the interval $ab$ is
  contained in at most finitely many members of $G$, then there is a finite
  subcollection of $G$ which covers interval $ab$.
\end{theorem}

\begin{problem}
  Let $G$ be a collection of regions that covers the interval $ab$.
  \begin{enumerate}[a.]
    \item Suppose the subcollection $G'$ of $G$ which covers \{a\} does not
      cover $ab$.  Prove that the set of right hand endpoints of members of
      $G'$ has a least upper bound $L$ which precedes $b$.
    \item Prove that the interval $aL$ is covered by a finite subcollection of
      $G$.
  \end{enumerate}
\end{problem}

\begin{question}
  Suppose $G$ is an infinite collection of regions covering the interval $ab$.
  Let $M$ be the set of all points $x$ in the interval $ab$ which (i) are not
  the point $a$ and (ii) finitely many members of $G$ cover the interval $ax$.
  \begin{enumerate}[a.]
    \item Is $M$ empty?  Prove your answer.
    \item Does $b$ belong to $M$? Prove your answer.
  \end{enumerate}
\end{question}

\begin{definition}
  A point set $M$ is said to be \textbf{compact} if and only if each open cover
  of $M$ contains a finite subcover of $M$.
\end{definition}

\begin{theorem}
  If $G$ is a collection of regions covering the interval $ab$, then there is
  some finite subcollection of $G$ that covers the interval $ab$.
\end{theorem}

\begin{theorem}
  Strengthen Theorem 49 by letting $G$ be an open cover.
\end{theorem}

\begin{theorem}
  Each closed and bounded point set of $M$ is compact.
\end{theorem}

\begin{theorem}
  Every bounded infinite point set has a limit point.
\end{theorem}

\begin{theorem}
  Every closed and bounded point set contains a first point and a last point.
\end{theorem}

\begin{theorem}
  If $M_1, M_2, M_3, \cdots$ is an infinite sequence of closed and bounded
  point sets such that for each $n$, $M_n$ contains $M_{n+1}$, then the sets
  have a point in common.  Furthermore, the set of all points common to these
  sets is closed.
\end{theorem}

\begin{theorem}
  If $M$ is an infinite set of points, then there is an increasing or a
  decreasing sequence of points in $M$.
\end{theorem}

\begin{definition}
  A point set $M$ is said to be \textbf{countable} if $M$ is either finite or
  $M = \{p_1, p_2, p_3, \cdots\}$ where $p_1, p_2, p_3, \cdots$ is an infinite
  sequence of distinct elements.
\end{definition}

\begin{problem}
  Give some examples of countable sets.
\end{problem}

\begin{problem}
  Show that the set $M = \{(m,n) | m \textrm{and} n \textrm{are positive
  integers}\}$ is countable.  Start by drawing a graph of $M$.
\end{problem}

\begin{problem}
  Show that the set of rational numbers is countable.
\end{problem}

\begin{problem}
  Show that the union of two disjoint infinite countable sets is countable.
\end{problem}

\begin{problem}
  Show that the union of countably many countable sets is countable.
\end{problem}

\begin{definition}
  A point set $M$ is said to be \textbf{uncountable} if $M$ is not countable.
\end{definition}

\begin{problem}
  Show that the assumption that the set of all sequences of zeros and ones is
  not countable leads to a contradiction.
\end{problem}

\begin{definition}
  A collection $G$ of sets is said to be \textbf{monotonic} if and only if
  given two members of $G$, one is a subset of the other.
\end{definition}

\begin{definition}
  The point set $M$ is said to be \textbf{perfectly compact} if and only if for
  each monotonic collection $Q$ of subsets of $M$, some point belongs to each
  member of $Q$ or some point is a limit point of each member of $Q$.
\end{definition}

\begin{theorem}
  Each perfectly compact point set is compact.
\end{theorem}

\begin{question}
  Is each compact set perfectly compact?
\end{question}

\begin{definition}
  A point set $M$ is said to be \textbf{perfect} if $M$ is closed and every
  point of $M$ is a limit point of $M$.
\end{definition}

\begin{theorem}
  No countable point set is perfect.
\end{theorem}

\begin{problem}
  The set of numbers is uncountable.
\end{problem}

\begin{theorem}
  Every region is uncountable.
\end{theorem}

\begin{theorem}
  Every uncountable point set has a limit point.
\end{theorem}

\begin{definition}
  A point set $M$ is said to be \textbf{nowhere dense} in a point set $K$ if
  every region intersecting $K$ contains a region that intersects $K$ but not
  $M$.
\end{definition}

\begin{theorem}
  If the point set $H$ is nowhere dense in the point set $K$, then every subset
  of $H$ is nowhere dense in $K$.
\end{theorem}

\begin{theorem}
  If each of the point sets $H$ and $K$ is nowhere dense in the point set $M$,
  then $H \cup K$ is nowhere dense in $M$.
\end{theorem}

\begin{theorem}
  No region is the union of a countable number of point sets such that each of
  them is nowhere dense in $S$.
\end{theorem}

\begin{theorem}
  No closed point set $M$ is the union of a countable number of closed point
  sets such that if $X$ is any one of them, then every point of $X$ is a limit
  point of $M - X$.
\end{theorem}

\begin{theorem}
  No closed point set $M$ is the union of a countable number of point sets such
  that each of them is nowhere dense in $M$.
\end{theorem}

\begin{definition}
  Suppose $K$ is a point set.  A point set $H$ is said to be \textbf{everywhere
  dense} in $K$ if and only if $H$ is a subset of $K$ and every point of $K$ is
  a point of $H$ or is a limit point of $H$.
\end{definition}

\begin{theorem}
  If the point set $H$ is everywhere dense in the point set $K$, then every
  region intersecting $K$ contains a point of $H$.
\end{theorem}

\end{document}

% vim: tw=79
