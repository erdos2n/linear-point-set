\documentclass{article}
\usepackage{amsthm}
\usepackage{enumerate}

\theoremstyle{definition}
\newtheorem{axiom}{Axiom}
\newtheorem{theorem}{Theorem}
\theoremstyle{definition}
\newtheorem{example}{Example}
\newtheorem{problem}{Problem}
\theoremstyle{plain}

\title{\textbf{\textsf{Linear Point Set Theory}}}
\date{}

\begin{document}
\maketitle

\section*{Order Properties}

\paragraph{Undefined Terms:} The word ``point'' and the expression ``the point
$x$ precedes the point $y$'' will not be defined.  This undefined expression
will be written $x < y$.

In these notes, all sets have at least one element.  By ``point $x$ follows the
point $y$'' we mean $y < x$.  By $x \leq y$ we mean that $x < y$ or $x = y$.

\begin{axiom}
  $S$ is a collection of points such that
  \begin{enumerate}[a.]
    \item if $x$ and $y$ are different points, then $x < y$ or $y < x$.
    \item if the point $x$ precedes the point $y$, then $x$ is not equal to $y$.
    \item if $x$, $y$, and $z$ are points such that $x < y$ and $y < z$, then $x < z$.
  \end{enumerate}

\end{axiom}

In case a misunderstanding persists concerning the relationship between
``point'', ``point set'', and $S$, let it be clear that, unless there is a
stipulation to the contrary, that

\begin{itemize}
  \item ``point'' always means a member of $S$.
  \item ``point set'' means a nonempty set of points in $S$ or a nonempty subset of $S$.
  \item $S$ is the universal set, that is, the set of all points.
\end{itemize}

\begin{example}
  Suppose by ``point'' we mean any integer and by ``$x$ precedes $y$'' we mean
  the normally understood ``$x$ is less than $y$.''  Clearly Axiom 1 is
  satisfied.
\end{example}

\begin{example}
  Suppose by ``point'' we mean a number depicted below and by ``$x$ precedes
  $y$'' we mean that $x$ is to the left sf $y$ on the number line.
\end{example}

\begin{example}
  Suppose by ``point'' we mean a number depicted below and by ``$x$ precedes
  $y$'' we mean that $x$ is to the left of $y$ on the number line.  See below:
  \[3 \qquad -2 \qquad 1 \qquad 5 \qquad -7.1 \qquad 7 \qquad 6\]
  Is this a model of Axiom 1?
\end{example}

\begin{problem}
  Think of a particular dictionary of English words.  Suppose by ``point'' we
  mean any word in that dictionary.  What interpretation could you give to
  ``precedes'' so that Axiom 1 is satisfied?
\end{problem}

\begin{problem}
  Consider the HFD Corporation, the world's largest corporation that hires
  thousands of people every year.  Suppose by ``point'' we mean an employee of
  HFD.  By ``$x$ precedes $y$'' we mean that employee $x$ was hired on at HFD
  before employee $y$.  Are Axioms 1a, 1b, and 1c all satisfied? Explain.
\end{problem}

\begin{problem}
  Find an interpretation for Axiom 1 such that Axiom 1a and 1c are satisfied,
  but 1b is not satisfied.
\end{problem}

\begin{problem}
  Give a model of Axiom 1 where ``point'' means integer and ``$x$ precedes
  $y$'' is defined in such a way that 0 precedes all other integers.
\end{problem}

When we have found meanings for the undefined terms of our axiom system such
that the axioms of our system are satisfied as we did in Examples 1 and 2
above, we have a \emph{model} of the axiom system.

\begin{problem}
  Give a meaning sf ``precedes'' where ``point'' means an ordered number pair
  and Axiom 1 is satisfied.  State what a typical region would look like.
\end{problem}

\begin{theorem}
  If $x$ and $y$ are different points such that $x < y$, then $y \not< x$.
\end{theorem}


\end{document}

% vim: tw=79
